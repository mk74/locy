Phone sensing can be utilized by mobile applications to provide advanced services such as navigation systems. Phone sensing fetches raw sensor data\ (e.g. from an accelerometer) and tries to extract high-level information from it\ (e.g. a user is walking). Such a process may have high energy demands, which is cruciallly important to mobile phone users. To alleviate this problem, energy-efficient  phone sensing methods must be investigated.  

This project investigates the energy-efficiency of phone sensing. It establishes the energy efficiency of all sensors available across three different devices. The results are used to design Locy, an energy-efficient localisation library for Anrdoid, which is easy to plug into existing applications. When possible, Locy replaces the high-power GPS receiver with readings from the energy-efficient accelerometer. Its algorithm adapts to the remaining battery life of a device. Locy's energy efficiency is evaluated in two real-life scenarios. The library provably outperforms a standard Android localisation implementation.

Additionally, a new method for energy efficiency measurement is introduced and verified. This is believed to be the first study presenting complete results of sensors' energy efficiency across different devices. The results may be further leveraged for different research purposes. Finally, Locy provides a GPS localization service in an energy-efficient manner, which may be of great interest to developers wishing to increase the utility of their mobile applications.

