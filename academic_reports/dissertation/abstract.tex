Phone sensing can be utilized by mobile applications to provide advanced services\ (e.g., navigation system).  Phone sensing fetches raw sensor data\ (e.g., accelerometer) and tries to extract high-level information out of it\ (e.g., a user is walking). The process may have high energy demands, which are of crucial importance to mobile phone users. To alleviate this problem, the energy-efficient methods for phone sensing have to be investigated.  

The project investigates the energy-efficiency of phone sensing. It establishes the energy efficiency of all the sensors available across three different devices. Those results are used to design the energy-efficient Android library, Locy. Locy provides GPS localization services and is easy to plug into existing applications. If possible, Locy replaces high-power GPS with energy-efficient accelerometer. Its algorithm is adaptive to the current battery life of a device. The energy efficiency of Locy was evaluated in two real-life scenarios. It was proven that the library outperforms a standard Android implementation. 

The project introduces and verifies a new method for energy measurements. The method can be used in other research. The complete results of sensors' energy efficiency across different devices is believed to be first study of this type. Those results may be further leveraged for different research purposes. Finally, Locy provides GPS localization service in an energy-efficient manner. The library may be of crucial important to developers to increase the utility of their mobile applications.
	
	