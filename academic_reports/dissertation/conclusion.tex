\section{Conclusions}
\label{s:conc}
Three areas relevant to this project have been reviewed\ (phone sensing, energy measurements methods and energy-efficient sensing). A possible enhancement has been identified in each area. It was concluded that further progress in phone sensing could be made by building a large sensor data set of different user activities. Furthermore, measuring 1\% of battery life was proposed as a suitable method of energy measurements for the project. Lastly, hybrid approaches of different energy-efficient techniques were suggested to achieve additional energy savings. 
Measuring 1\% of battery life is a novel way of investigating the energy efficiency of a mobile phone. The method was tested in a series of the experiments and found to be valid. It also revealed specific characteristics of battery life\ (wrong samples and energy efficiency levels). However, the method is unstable for older batteries, and thus unsuitable for online energy measurements. 		

The series of experiments yielded the energy efficiency of all sensors available across three different devices. It is believed to be the first study of this type. Different sensors had different energy demands depending on device, pointing out the importance of online energy measurement for energy-efficient sensing. The energy efficiency of the sensors utilized for localization were analyzed in more detail. The results explain how localization technologies evolve. Finally, physical sensors turned out be more energy-efficient than standard localization sensors\ (GPS and IEEE 802.11 scanning). This observation was the basis for the design of the energy-efficient sensing library, Locy.

Locy provides a GPS localization service in an energy-efficient manner, by leverating the energy-efficient accelerometer to infer whether a user is moving and hence whether it's necessary to invoke the high-power GPS at all. Further energy optimization is achieved by utilizing duty-cycling when samplingthe accelerometer. The duty-cycling ratio\ (sampling time over sleeping interval) is changed depending on the level of battery life, using less energy if the battery is low. Locy can easily be plugged into existing applications since it exposes the same API as the LocalizationManager provided by Android. Plugging Locy into Tristan's ESM was a one-liner.

Locy's energy efficiency was evaluated in a separate series of experiments. Its performance was compared against the naive GPS implementation. The experiments involved two real-life scenarios conducted on three different devices. Locy turned out to have lower energy demands than the naive GPS implementation. 

Two extensions for Locy were proposed: inLocy\ (Wi-Fi based localization) and Locy working as a service. In future work, these extensions could be further developed and evaluated in real-life scenarios. Locy could also leverage other ideas for energy-efficient sensing: The regular patterns in human behaviour may help optimize duty-cycling when sampling\ (by changing duty-cycling ratio accordingly). A next research project is planned to further investigate that idea.
		
This project extended knowledge in energy-efficient phone sensing. It revealed novel characteristics of battery life and provided the complete set of sensor energy efficiencies across different phones. An energy-efficient sensing library was created, outperforming the standard Android implementation. 
	
