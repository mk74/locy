\section{Conclusions}
\label{s:conc}
Three areas which are relevant to this project has been reviewed\ (phone sensing, energy measurements methods and energy-efficient sensing). In each of those areas, a possible enhancement has been identified. It was concluded that phone sensing could be further progressed by building a large sensor data set of different user activities. Furthermore, measuring 1\% of battery life was proposed as the suitable method of energy measurements for the project. Lastly, the hybrid approaches of different energy-efficient techniques were suggested to achieve additional energy savings. 

Measuring 1\% of battery life is a novel way of investigating the energy efficiency of a mobile phone. The method was tested in the series of the experiments and turned out to be valid. It also revealed specific characteristics of battery live\ (wrong samples and energy efficiency levels). However, the method is not stable for older batteries, and thus, it cannot be used for online energy measurements. 		

The series of the experiments provided the energy efficiency of all sensors available across three different devices. It is believed to be the first study of this type. There were different sensors' energy demands depending on a device, which justifies why online energy measurement is important for the energy-efficient sensing. The energy efficiency of the sensors utilized for localization were analyzed in more details. Those results explain how localization technologies evolve. Finally, physical sensors turned out be more energy-efficient than standard localization sensors\ (GPS and IEEE 802.11 scanning). This observation was used for the design of the energy-efficient sensing library, Locy.

Locy provides GPS localization service in an energy-efficient manner. It leverages energy-efficient accelerometer to infer whether a user is moving. If a user is not moving, high-power GPS can be switched off to save energy.  For further energy optimization, it utilizes duty-cycling accelerometer sampling. The ratio of duty-cycling\ (sampling time over sleeping interval) changes depending on the level of battery life. As a result, the library uses less energy if the battery life is low. Locy can be easily plugged to existing applications since it exposes the same API as LocalizationManager provided by Android. The replacement of line was needed to plug Locy into Tristan's ESM.

The energy efficiency of Locy was evaluated in the separate series of the experiments. Its performance was compared against the naive GPS implementation. The experiments involved two real-life scenarios and was conducted over three different devices. Locy turned out to have energy demands than the naive GPS implementation. 

Two extensions for Locy were proposed: inLocy\ (Wi-Fi based localization) and Locy working as a service . In future work, those extensions could be further developed and evaluated in real-life scenarios. Also, Locy could leverage other ideas for energy-efficient sensing. The regular patterns in human behaviour may help in optimizing duty-cycling sampling\ (by changing duty-cycling ratio accordinglt). The next research project is planned to further investigate that idea.
		
The project extended knowledge in energy-efficient phone sensing. It revealed novel characteristics of battery life and provided the complete set of sensor's energy efficiency across different phones. In the project, the energy-efficient sensing sensing library was created. The library was evaluated and outperforms the standard Android implementation. 
	