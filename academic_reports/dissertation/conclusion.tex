\section{Conclusions}
\label{s:conc}


Sensor Energy measurement's method:
	-design novel way of investigating energy efficiency
	-checked it in practice, identify its problems
	-the method is validated as the results are compared with other research
	-cons:
		-the method is instable for older batteries 
			-and thus can only be used in limited environment: no online measurement
	
	
Sensor Energy measurement's results:
	-complete set of results across three different phones including all sensors
		-it is believed that the first one like this
	-different energy characteristics across different devices
		->reason for online measurement, invalidate power models
	-the evolution of sensors for localization services
	-cons:
		-no different parameters for phones are checked:
			-no duty cycling checked, no different periods/sleeping intervals checked
	-those results are used to design the energy-efficient algorithms for the library:
		-sensor substitution as cheaper sensor may be leveraged to replace heavy-duty sensors
		-results also raise questions: "combined sensors" -> how they are going to perform?
	
The energy-efficient library:






Future work as well