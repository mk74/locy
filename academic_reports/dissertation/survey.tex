\section{Context survey}
\label{s:contextsurvey}
\subsection{Energy measurement}
\hspace{10pt} To improve energy-efficiency of sensing, an accurate method of measuring power consumption needs to be established. This task is relatively easy for Nokia phones. Nokia Energy Profiler \cite{nokia:profiler}, an application delivered by Nokia,  accurately measures power consumption and is widely used by research community \cite{kjaergaard:entracked} \cite{lu:jigsaw} \cite{li:status}. However, Google does not provide an equivalent application for Android smartphones, which makes the task difficult.

The Power Monitor is a hardware power meter delivered by MoonSooon PowerSolutions \cite{monsoon:powermonitor}.  It is designed for analyzing power of single lithium\ (Li) batteries (all Android phones). Although the tool provides accurate measurements, it can only be used in lab environment e.g., it can't measure energy consumption of GPS sampling while a client is moving outdoor. It is also expensive, and thus, it can't be directly used by mobile phone clients. To easy this problem, mobile phone vendors establish power profiles and provide them on mobile phones \cite{android:powerprofiles}. This information may be further utilized by software power profilers.

Software power profilers on Android smartphones use a power model to determine current energy consumption. They continuously query states of hardware components such as CPU, wireless card or screen. For example, wireless card may be switched off, listen or send information. Each state of every component has its corresponding energy consumption. This value is available in power profiles provided by a mobile phone vendor as described in the previous paragraph. Software power profilers aggregate all of those energy values to estimate current energy consumption of a device. Example applications using this method includes PowerTutor \cite{zhang:powertutor} and Battery Monitor Widget \cite{googleplay:batterymonitorwidget}. Although this method provides online energy measurement and it can by used by customers, those measurements are often inaccurate. With time, the power profiles, provided by vendors, can change as battery characteristics do [XXX references]. To easy this problem, PowerTutor adopts its algorithm basing on the device model. However, it only supports three mobile phones: HTC G1, HTC G2 and Nexus one. Furthermore, the power profilers are not applicable to phone sensing: power profiles are often not available for physical sensors e.g., PowerTutor does not support accelerometer nor camera.

[XXX some better name than Software power profilers would be useful!
 maybe battery consumption measurement]\\

Battery life as metrics\\
	BetterBatteryStats\\
	Battery Stats Plus\\
	research community: SenseLess\cite{benabdesslem:senseless}\\
	cons:\\
		argument from Narseo's paper\\
		take long\\
			maybe don't measure full depletion
	
	
	
SOFTWARE AND HARDWARE Together\\
	TREPN PROFILER\cite{qualcomm:trepnprofiler}\\
		QualComm\\
		how it works??\\
			"baseline computations and then delta difference"\\	
			low level software running on specific chip-set\\
				this software and chip-sets the same company produces\\
		critical thinking:\\
			relatively accurate\\
				(repetitive results where conditions the same)\\
			but\\
				only work on specific hardware \\
				still in progress\\
		
WHERE?\\
	Android power consumption values when i list all sensors?\\
		validate them?\\

CURRENT ENERGY RESULTS\\
	Senseless\cite{benabdesslem:senseless}\\
	

CONCLUSIONS\\
	...\\
	TrepnProfiler very interesting and that may be a future\\
	measuring battery life:\\
		 don't measure full depletion\\


\subsection{Phone sensing}
ACTIVITY RECOGNITION\\
	Google Activity recognition part of Google Play \cite{android:activityrecognition}\\
		used by Google Maps\\
		part of Google Play, which works while customer is online\\

INERTIAL NAVIGATION\\
	navigation usually done by GPS or Wireless communication\\
	but uses more energy and is often not precise enough(for indoor localization)\\
		(some reference to geofencing)\\
	so improve its accuracy/energy consumption by inertial navigation\\
	Towards Mobile Phone Localization without War-Driving\cite{constandache:localization}\\
	GAC: Energy efficient hybrid GPS-Accelerometer-Compass GSM location\\
	Indoor\\
		growing organic..\\
		unLoc\\
		stuff from Valentin Radu\\
	
SLAM\\

Localization \\
	SurroundSense \cite{azizyan:surroundsense}\\

Hardware improvements\\
	Samsung Galaxy 5 -> step counter, step detector\\
	Project Tango - rich sensors, making many difficult tasks possible\cite{google:tango}\\

CONCLUSIONS\\
	progress?\\
	Collect Big Data\\
		was already proposed Hossmann et al.\cite{hossmann:bigdatasets}\\
		something like Reality Mining \cite{eagle:realitymining}\\
	
\subsection{Energy-accuracy tradeoff in phone sensing}
hierarchy/temporal relations\\
	Fused Location provider\cite{android:locationapi}\\
		part of Location APIs, uses Google Play(activity recognition)\\
	EEMSS \cite{wang:eemss}\\
	senseless \cite{benabdesslem:senseless}\\
	ACE \cite{nath:ace}\\

adaptive sampling\\
	Rachuri et al. \cite{rachuri:dynamicsensing}\\
	SociableSense \cite{rachuri:socialsense} \\
	similar idea in other fields:\\
		wireless:\\
			channel sampling \cite{deshpande:channeling}\\
			refocusing \cite{deshpande:refocusing}\\
			coordinated sampling \cite{deshpande:coordinated}\\

offloading to the cloud\\
	Uploading strategies \cite{musolesi:offloading}\\
	SociableSense \cite{rachuri:socialsense} \\
	
two-tier\\
	Accurate two-tier classifier \cite{srinivasan:twotier}\\

machine learning\\
	Enloc \cite{constandache:enloc}\\
	SmartDC \cite{chon:smartdc}\\

others\\
	this checking IPC thing\\

conclusions
