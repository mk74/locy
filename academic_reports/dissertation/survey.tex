\section{Context survey}
\label{s:contextsurvey}
\hspace{10pt} This section critically reviews the project's related work\ (Objective \ref{o:survey:review}). First, it studies the capabilities of phone sensing. Second, it explores different energy measurement methods and presents other research's energy measurement results. Finally, the approaches to energy-efficient phone sensing are studied. The section also tries to identify possible improvements in each of those areas\ (Objective \ref{o:survey:enhancements}).

\subsection{Phone sensing}
\hspace{10pt} Although physical sensors are noisy and often inaccurate, meaningful information may be extracted from them. Sensor-based \textbf{activity recognition} aims to determine a user's physical activity e.g. sitting, walking or driving. Most of this activity recognition is based on accelerometer data, though other sensors may be utilized\ (e.g. compass). As a part of Google Play Services, an activity recognition service is provided in Android: AcitivityRecognitionClient \cite{android:activityrecognition}. At the moment, the service is capable of detecting five different types of activities: 
 \begin{itemize}
  	\item a user is in vehicle.
    \item a user is on bike.
    \item a user is on foot.
    \item the user is still.
    \item the mobile phone is tilting.
  \end{itemize}
  
ActivityRecognitionClient delivers the list of activities with the corresponding probabilities of them taking place at a given moment. The service requires Internet connectivity to work, since Google Play Services do. Furthermore, the research community has also established algorithms for detecting other activities:
 \begin{itemize}
  	\item a user is taking an elevator up or down \cite{Wang:unsupervised}.
    \item a user is ascending or descending a staircase \cite{Wang:unsupervised}.
    \item a user is on escalator \cite{Wang:unsupervised}.
    \item a user is running \cite{miluzzo:cenceme}.
    \item a different position of a phone\ (e.g., chest pocket or trousers pocket) \cite{kawahara:positions}.
  \end{itemize}

Activity recognition algorithms are still in their early days. Some studies propose different algorithms for recognizing the same activity\ (e.g., walking \cite{benabdesslem:senseless} \cite{rachuri:socialsense} \cite{Kwapisz2011} ). Furthermore, implementations of the proposed algorithms arbitrarily select different parameters: threshold values\ (e.g., if a standard deviation is more than 30, then a user is walking) and window size\ (e.g., classify weather a user is moving over the period of two seconds). Lastly, the algorithms are usually evaluated on a limited amount of devices with few users, who hold their phones in constrained positions (e.g. trouser pocket). 

Activity recognition may be leveraged for \textbf{navigation}. Detected activities may assist in localizing a user e.g., if a user is taking an elevator and there is one elevator nearby, he can be easily localized. Furthermore, physical sensors may provide useful information for navigation: the speed of user's movement\ (accelerometer) and their direction\ (compass and gyroscope). Such information may be utilized to estimate a user's relative position. For example, take a user moving north in straight line with a speed of 5 meters per second. After a minute, his position is estimated to be 100 meters north in a straight line from the previous position.  This process is called Pedestrian Dead Reckoning\ (PDR) and is often combined with absolute localization systems\ (GPS or Wi-Fi based localization). In GAC \cite{youssef:gac}, infrequent GPS samples calibrate a user's relative position calculated by PDR (accelerometer and compass data). An absolute localization system may also be replaced by other mechanisms. Constandache et al.\cite{constandache:localization} propose an inertial navigation system which does not rely on GPS nor Wi-Fi based localization. It instead leverages electronic maps, which are equipped with additional information on how sensors readings should be changing while a user is taking a specific path. The system uses directional information\ (compass and gyroscope) to identify a path. For example, a user may be localized to be on a curved path if his compass' readings steadily change in the same direction. Directional information for maps may be automatically generated and pre-fetched to the mobile phone.

Inertial navigation is further leveraged for \textbf{indoor localization}. UnLoc \cite{Wang:unsupervised} identifies certain locations based on their sensors' signatures\ (also SensLoc \cite{kim:sensloc} ). For example, the elevator has a specific accelerometer signature or a computer lab may create a characteristic magnetic field. After those landmarks with their fingerprints are learned by the system in an unsupervised manner, they enable a user to be localised if they return to those places. A similar idea has been previously applied in Wi-Fi based indoor localization systems e.g., Park et al.\cite{park:organic}. The hybrid solution, which combines Wi-Fi fingerprints and other sensor signatures has been also suggested by Radu et al.\cite{radu:hybrid}. Lastly, SurroundSense \cite{azizyan:surroundsense} leverages the ambient sound as a localization fingerprint.

Another idea to improve the capabilities of phone sensing is placing \textbf{high resolution sensors} on a device, which may provide more detailed readings. Furthermore, more sensors are put on a mobile phone to infer the context more accurately. Google is currently working on Project Tango \cite{google:tango}, which materializes both of those ideas. The prototype of that project can localize a user and build the map of the environment\ (simultaneous localization and mapping) based on many mobile phones' high resolution sensors and camera. 		
		
Although sensors provide much information, it is very difficult to extract meaningful information from them. There are many practical challenges for activity recognition\ (different algorithms for the same activity, their various parameters and their generality). They also affect other phone sensing applications e.g., inertial navigation. It is believed that those challenges may be addressed by comparing and evaluating algorithms against a big sensor data set of different activities. This idea has been proposed before by Hossmann et al.\cite{hossmann:bigdatasets} and successfully executed in other fields e.g. Reality Mining \cite{eagle:realitymining}. In phone sensing, Kwapisz et al. \cite{Kwapisz2011} is believed to have collected the largest data set\ (29 different users). However, the all data collection was constrained to one phone position: a front trouser leg pocket. 
		
\subsection{Energy measurement}
\hspace{10pt} To improve the energy efficiency of phone sensing, an accurate method for measuring it needs to be established. This task is relatively easy for Nokia phones. Nokia Energy Profiler \cite{nokia:profiler}, an application delivered by Nokia,  accurately measures power consumption and is widely used by research community \cite{kjaergaard:entracked} \cite{lu:jigsaw} \cite{li:status}. However, Google does not provide any equivalent application for Android smartphones. This makes the task of energy measurement difficult. The first subsection reviews the methods for measuring power consumption available on Android and the second subsection demonstrates energy measurement results of other studies.

\subsubsection{Energy measurement methods}
\hspace{10pt} \textbf{The Power Monitor} is a hardware power meter delivered by MoonSooon PowerSolutions \cite{monsoon:powermonitor}.  It is designed for analyzing power of single lithium\ (Li) batteries\ (all Android phones). Although the tool provides accurate measurements, it can only be used in a lab environment i.e., it cannot measure energy consumption of GPS sampling while a user is moving outdoors. Since the device is a costly\ (771 dollars), it cannot be directly used by most mobile phone users. To alleviate this problem, a mobile phone vendor establishes the power profile of a device as part of their manufacturing process. The power profile is then placed onto a mobile device \cite{android:powerprofiles} and may be further utilized by any battery usage application.

Most battery usage applications on Android smartphones leverage a \textbf{power model} to determine the device's current energy consumption. Those applications continuously query the state of hardware components such as CPU, wireless card or display to estimate their energy demands. For example, the wireless card may be switched off/on or scanning. Each of those states has an assigned energy consumption e.g. if a device is switched off, its power demand is 2 mA power; if switched on, its power demand is 31 mA and while scanning, its power consumption is 100 mA. All such energy consumption values are available in the power profile provided by a mobile phone vendor (as described in the previous paragraph). The applications aggregate energy consumption values of current states of all components to estimate the total power consumption of a device. Those applications include PowerTutor \cite{zhang:powertutor} and Battery Monitor Widget \cite{googleplay:batterymonitorwidget}. Although they provide online energy measurements and can by used by users, their measurements are often inaccurate. With time, the power profile provided by a vendor can change, as the battery itself does. To ease this problem, PowerTutor adapts its algorithm based on device model. However, it only supports three mobile phones: HTC G1, HTC G2 and Nexus one. Finally, the battery usage applications based on a power model are not applicable to phone sensing: the power profiles provide no information on power consumption of physical sensors i.e., PowerTutor does not support accelerometer.

\textbf{Battery life} may be also measured to estimate the power consumption of a device. If a mobile device has high energy demands, its battery life should deplete quickly. BetterBatteryStats \cite{googleplay:betterbatterystats} and Battery Stats Plus \cite{googleplay:batterystatsplus} use battery life as a metric to monitor mobile applications' energy consumption. This idea has also been utilized by the research community: SenseLess \cite{benabdesslem:senseless} compares the energy efficiency of different sensors based on how long the full battery depletion takes while only one sensor is being sampled. This approach provides an accurate approximation of the power consumption of a device and does requires no additional hardware. Furthermore, the method may be applied by mobile phone users and also works for physical sensors. However, the measurements may be time-consuming. In SenseLess, it took 170 hours to fully deplete the battery life with all sensors switched off. Since 2009\ (Senseless), the battery capacity has increased and the Android operating system has become more energy efficient. Full battery depletion is likely to take even longer for modern mobile phones.

\textbf{TrepnProfiler} \cite{qualcomm:trepnprofiler} is a new method for measuring power consumption of a device. The tool, provided by Qualcomm, is a low level software, which runs only on SnapDragon chips, which are also provided by Qualcomm. The tool provides accurate measurements, including physical sensors, and may be used by mobile phone users. However, the tool is still relatively new and works only on mobile phones with SnapDragon chips.

\begin{table}[H]
	\centering
    \begin{tabular}{| c | c | c | c | c | c | c |}
    \hline
    Method & Accurate & Duration & \begin{tabular}[x]{@{}c@{}}Physical \\ Sensors\end{tabular} &   \begin{tabular}[x]{@{}c@{}}All mobile\\ phones\end{tabular} & \begin{tabular}[x]{@{}c@{}}Outside \\ lab environment\end{tabular} \\ \hline
    The Power Monitor & \checkmark & Immediate & \checkmark &  \checkmark  & -\\ \hline
    Power model & - & Immediate & - & \checkmark & \checkmark \\ 
    \hline
     \textbf{Battery life} & \ding{51} & \textbf{Long} & \ding{51}  & \ding{51}  & \ding{51}  \\ 
     \hline
    TrepnProfiler & \checkmark & Immediate & \checkmark & - & \checkmark \\ \hline
    \end{tabular}
    \caption{The comparison of energy measurement methods. Using battery life as metrics is the most suitable method for the project.}
	\label{table:energymeasurementmethods}
\end{table}
	
In this subsection, four different energy measurement methods were reviewed. The characteristics of them are summarized in the Table \ref{table:energymeasurementmethods}. The project requires accurate energy measurements outside of the lab environment\ (i.e. studying power consumption of a device while a user is walking outdoors). The method of \textbf{measuring battery life seems to be the most suitable} for that purpose. The method has one disadvantage: it is time-consuming. To alleviate this problem, the 1\% percentage battery depletion is proposed instead of the full battery depletion. This approach is further explained in Section \ref{s:design:measurements:method}. TrepnProfiler would also satisfy the project's requirements, but the tool does not work on all Android devices. However, SnapDragon chips are becoming a standard in the industry and the tool itself may become one as well. 

\subsubsection{Energy measurement results}
\hspace{10pt} The energy efficiency of sensors has been examined before \cite{benabdesslem:senseless} \cite{constandache:localization} \cite{wang:eemss} \cite{chon:smartdc}.  Those studies provide similar results regardless of the energy measurement methods used; e.g. the \plotref{senseless} shows the energy consumption of different sensors established in SenseLess \cite{benabdesslem:senseless}. All of the studies confirm that the accelerometer is more energy-efficient than the microphone, which consumes less power than GPS and IEEE 802.11 scanning, which are more energy-efficient than the video camera.

\plot{senseless}

Depending on the energy measurement method used in other studies, they provide more details on sensors' energy consumption. For example, GAC \cite{youssef:gac}, which utilizes the Power Monitor device as its measurement method, establishes the GPS power consumption pattern over time\ (i.e. the peaks of energy consumption at the moments when samples are taking places i.e. every 20 seconds). Also, the energy consumption of sensors may differ depending on their parameters. SmartDC \cite{chon:smartdc} studies how power consumption of GPS and Wi-Fi scanning changes depending on their sampling interval i.e. higher sampling frequency results in higher power consumption. SmartDC also investigates the energy demands of an accelerometer depending on its duty-cycling ratio\ (sampling time over sleeping interval). A smaller ratio\ (i.e. small sampling time relative to sleeping interval) results in less power consumption.  

All of those studies either only focus on few sensors (e.g. accelerometer, GPS and IEEE 802.11 in SmartDC \cite{chon:smartdc}) or measure sensors' power consumption on only one device \cite{benabdesslem:senseless}. To validate those results, this project aims to provide the complete set of all sensors' energy measurements across three different devices. 
	
\subsection{Energy-efficient phone sensing}
\hspace{10pt} \textbf{Relations between sensors} may be exploited for energy-efficient phone sensing. For example, if a user is stationary\ (it can be detected by accelerometer), their geographical position\ (GPS coordinates) will not change. This relation between accelerometer and GPS, allows the high-power sensor\ (GPS) to be switched off based on the results of the energy-efficient one (accelerometer). This observation is used by the Fused Location Provider \cite{android:locationapi}, a smart location framework provided by Android. In the research community, SenseLess \cite{benabdesslem:senseless} leverages the same idea of substituting GPS with accelerometer. There are also frameworks that utilize more complex relations between sensors. EEMSS \cite{wang:eemss} recognizes high-level user state (e.g., working in a office) and optimizes by learning relations between motion information\ (accelerometer), location\ (GPS and Wi-Fi based) and background noise\ (microphone). If it is quiet and an user is not moving, he is working in a office or resting at home\ (no need to sample GPS). ACE \cite{nath:ace} is a similar system, which is also capable of making guesses about user state. Additionally, it also acts as a middleware, which may save energy among many applications.

Continuous sampling of physical sensors\ (e.g. accelerometer) is energy inefficient. Since all high-power hardware components need to stay awake\ (LittleRock \cite{priyantha:littlerock}), it results in fast battery depletion. To reduce those energy demands, a sensor may be sampled only a fraction of the time e.g. two seconds of sampling and two seconds of not sampling\ (sleeping interval). Such an approach is called \textbf{duty-cycling}. Energy is saved, but interesting events\ (a user starts walking) may be missed if they happen over the sleeping intervals. There is a specific trade-off in choosing the duration of sleeping intervals. Longer sleeping intervals will save more energy, but increase the probability of missing interesting events. To resolve this, the duration of sleeping intervals may be adjusted depending on whether interesting events are being detected. For example, if a user starts walking, the accelerometer sampling should become more aggressive. This adaptive sampling methodology was proposed by Rachuri et al. \cite{rachuri:dynamicsensing}. SociableSense \cite{rachuri:socialsense} further develops this methodology.  It controls the duration of sleeping intervals depending on the history of detected events. If someone said one word, the duration of microphone sleeping intervals should be decreased, but not to the same extent as when continuous speech is detected over last half a minute. This idea is formalized as a learning technique based on the theory of learning automata. Similar ideas have also been previously studied in wireless communication \cite{deshpande:channeling} \cite{deshpande:refocusing} \cite{deshpande:coordinated}.

In phone sensing, feature extraction and classification phases are computation-intensive. It has been proposed to \textbf{offload the computations to the cloud}. This could result in energy optimizations, but also poses additional challenges: it affects latency\ (network delay) and a user's data plan\ (network bandwidth). SociableSense \cite{rachuri:socialsense} investigates the trade-off between energy consumption, latency and a user's data plan in more etails. Also, internet connectivity may not be always available or there may be poor wireless coverage. Musolesi et al. \cite{musolesi:offloading} studies the energy efficiency of different strategies depending on the type of Internet connectivity available. 

There are \textbf{regular patterns in human behavior and mobility} \cite{falaki:diversity} \cite{banerjee:batteries}, which could be leveraged for energy optimization in phone sensing. Those patterns may be learned by machine learning algorithms and may be utilized to schedule sensing more efficiently. EnLoc \cite{constandache:enloc} leverages the logical mobility tree to schedule GPS sampling. The tree holds information on when a user moves and where he goes e.g. he leaves his home at 8:00 am to get to the office. The walk takes him 20 minutes. EnLoc may learn this pattern and schedule GPS at 8:00 to confirm that a user left home as usual. Next, it will schedule GPS at 8:20 to verify that a user is in the office. Thanks to the user's regular pattern, GPS sampling conserves energy. SmartDC \cite{chon:smartdc} takes this concept further: it is capable of learning a user's mobility tree in an unsupervised manner.

\textbf{Other approaches} to energy-efficient phone sensing also exist. For example, Srinivasan et al. \cite{srinivasan:twotier} proposes two-tier activity classification. First, it performs short accelerometer sampling to determine whether a user is moving at all. If so, it samples accelerometer longer to establish exact user's physical activity. If a user is not moving, the second longer sampling is not performed. The total energy consumption of recognizing a user not moving is lower for two-tier activity recognition, therefore making the whole system more energy-efficient. Another interesting approach to energy-efficiency is Hsieh et al. \cite{hsieh:ipc}. It analyzes different Android Inter-Process Communication mechanisms for continuous sensing applications. Three mechanisms\ (Intent, Binder and Context Provider) are studied in terms of their latency, memory footprint and CPU usage. Although power consumption is not analyzed, memory footprint and CPU usage are likely to significantly impact energy efficiency. For payload sizes smaller than 4KB\ (as for all physical sensors), the Binder delivers the best performance with the smallest memory footprint and CPU usage.

Many different approaches to energy-efficient phone sensing have been proposed. As those approaches are focused on different parts of phone sensing, they could be combined to obtain higher energy efficiency. For example, regular patterns in human behavior could be incorporated into adaptive sampling. SociableSense \cite{rachuri:socialsense} uses a simple learning technique to control the sleeping interval duration. This could be more efficient if the regular patterns of a user were taken into account. For example, take a user who walks for 20 minutes at 8am every Monday. The duration of sleeping interval in accelerometer sampling could be adjusted more significantly on Monday than other days. Another example is leveraging two-tier activity classification for adaptive sampling: If a system is unsure how long sleeping intervals should be, it can sample for shorter period of time to figure it out. 

\subsection{Conclusions}
\hspace{10pt} In this section, the capabilities of phone sensing have been reviewed. It was concluded that phone sensing may be further improved by collecting large data sets of labeled activities. In the subsequent subsection, energy measurements methods were studied and a suitable method for this project was chosen. Measuring time of 1\% battery depletion was proposed as an improvement to full battery depletion. Furthermore, the energy efficiency of sensors established by other studies was analyzed. Finally, approaches to energy-efficient sensing were studied. It was suggested that further energy optimizations may be achieved by combining different approaches e.g. leveraging machine learning and two-tier classification for adaptive sensor sampling. The section has addressed two primary objectives \ref{o:survey:review} and \ref{o:survey:enhancements} of the project.
