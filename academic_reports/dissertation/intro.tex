\section{Introduction}
\label{s:intro}
\hspace{10pt} Smartphones are widely used devices, which provide \textbf{advanced services based on a user's context}. Localization is the most utilized context information: Google Maps \cite{google:maps} and Apple Maps \cite{apple:maps} are popular navigation systems exploiting user's geographical coordinates; Yelp \cite{yelp:yelp} recommends a restaurant in a user's nearby; Facebook \cite{facebook:facebook} tags pictures with a user's geolocation and Foursquare \cite{foursquare:foursquare} is itself location-based social media. Beside localization, there are other relevant context information. Recently, the Apple's patent was granted for variable device graphical user interface \cite{uspatent:apple}. The patent describes the interface that is adjusted depending on what kind of phone motion is detected e.g. icons may be bigger while a user is running. Motion detection is also utilized in health mobile applications. Accupedo Pedometer \cite{googleplay:accupedo} monitors a user's daily walking to estimate his calories burned. Furthermore, other context information is widely used in research community e.g. speech detection and Bluetooth proximity may help in quantifying social interactions \cite{rachuri:socialsense}. Lastly, the context is likely to redefine mobile operating systems. The functionality of a mobile phone may be adaptive depending on a user's context e.g. a user may not be interested in bus timetable if he is driving a car. Google Now \cite{google:googlenow} tries to leverage that context information to provide a user with "the right information at just the right time".

\textbf{The phone sensing process infers the context}. The process consists of acquiring raw sensor data and extracting a relevant information out of them. For example, accelerometer data may indicate that the user is not moving. Modern mobile phones are rich in sensors. Google Nexus 7 is not only equipped with standard sensors such as microphone, camera, IEEE 802.11, GPS, Bluetooth or accelerometer, but also magnetic field, gyroscope or ambient light. All of those sensors may be sampled and provide raw sensor data from which meaningful information could be obtained. This process usually involves feature extraction and classification. Different features may be useful depending on a sensor and a type of information we want to obtain e.g., the number of peaks in accelerometer data may be a good predictor to answer whether an user is walking or running. The classification is an inference, which assigns a high-level class to feature vector e.g., 2 peaks in one second is interpreted as a user is walking.

\textbf{Battery life is of key importance to mobile phone users}. Modern mobile applications provide complex functionality, which results in high energy demands. Without any significant progress in battery hardware technology, energy consumption is one of the most crucial problems in mobile systems. Furthermore, there are global projects aspiring to provide mobile internet in developing countries e.g. Internet.org \cite{facebook:internetorg}. For them, the energy efficiency of a mobile phone is even more disturbing issue since the opportunity of charging a device is often limited.

\textbf{Phone sensing may lead to fast battery depletion}. Both phases of phone sensing incur additional energy costs. Simple physical sensor are energy-efficient themselves, but their careless sampling may result in high energy consumption. For example, while accelerometer is being sampled, high-power components including CPU are active which may significantly drain the battery life \cite{priyantha:littlerock}. For other sensors like camera or GPS, the sampling itself is an expensive operation \cite{benabdesslem:senseless}. Also, the data extraction process is computationally intensive, which means high energy demand \cite{musolesi:offloading}.

\textbf{In mobile phone sensing, there is a specific energy-accuracy tradeoff.} Energy efficiency of mobile sensing may be improved while decreasing the accuracy of the context information. For example, we could sense GPS less often. This would result in energy savings, but the location coordinates could be inaccurate during the periods when GPS is not being sampled.

\textbf{This project investigates an energy-accuracy tradeoff in phone sensing}. It proposes an unique energy measurement method. The series of experiments validates this method and provides the energy efficiency characteristics of all sensors available across three different devices. It is believed to be a first study of this type. Basing on that study, the energy-efficient localization library is built, Locy.  The library's performance is evaluated in three real-life scenarios. 

The dissertation is structured in the following way: First, the objectives are explicitly stated\ (Section \ref{s:objectives}). In Section \ref{s:contextsurvey}, the current state of the art in sensing, energy measurement and energy-efficient sensing is reviewed and possible enhancements are identified. Then, software requirements and software engineering processes are described\ (Sections \ref{s:requirements} and \ref{s:processes}) . Section \ref{s:ethics} presents the short description of ethical considerations. Next sections are split into two threads: Sensor Energy Measurements and Locy. The former is focused on determining the energy efficiency profiles, whereas the latter presents the energy efficient sensing library.  Their design, implementation and evaluation are presented in Section \ref{s:design}, \ref{s:implementation} and \ref{s:evaluation}. The project's conclusions are presented in the last section \ref{s:conclusions}.
