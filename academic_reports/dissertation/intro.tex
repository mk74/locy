\section{Introduction}
\label{s:intro}
\hspace{10pt} Smartphones are widely used devices. They provide advanced services by exploiting user's context e.g., a client is indoor. The context is inferred in mobile sensing process. Although useful, mobile sensing may lead to fast battery depletion, which is of key importance for mobile phones. 

\textbf{Mobile phones sensing is a complex process.} It compromises acquiring raw sensor data and extracting a relevant information out of them. For example, accelerometer data may indicate that the user is not moving. Modern mobile phones are rich in sensors. Google Nexus 7 is not only equipped with standard sensors such as microphone, camera, IEEE 802.11, GPS, Bluetooth or accelerometer, but also magnetic field, gyroscope or ambient light. All of those sensors may be sampled and provide raw sensor data from which meaningful information could be obtained. This process usually involves feature extraction and classification. Different features may be useful depending on a sensor and a type of information we want to obtain e.g., number of peaks in accelerometer data may be a good predictor to answer whether an user is walking or running. The classification is an inference, which assigns a high-level class to feature vector e.g., 2 peaks in one second is interpreted as a client is walking.

\textbf{Energy consumption is a crucial issue in phone sensing.} Both phases of phone sensing incur additional energy costs. Simple physical sensor are energy-efficient themselves, but their careless sampling may result in high energy consumption. For example, while accelerometer is being sampled, high-power components including CPU are active which may significantly drain the battery life \cite{priyantha:littlerock}. For other sensors like camera or GPS, the sampling itself is an expensive operation [XXX example/reference]. Also, the data extraction process is computationally intensive, which means high energy demand. [XXX example/reference]. 

\textbf{In mobile phone sensing, there is a specific energy-accuracy tradeoff.} Energy efficiency of mobile sensing may be improved while decreasing the accuracy of the context information. For example, we could sense GPS less often. Although this results in energy savings, the location coordinates may be inaccurate during the periods when GPS not being sensed. 

\textbf{This project investigates an energy-accuracy tradeoff in phone sensing}. It reviews the current state of the art and identifies possible enhacements. Also, the unique method of energy measurement is proposed and used to determine energy profiles of different sensors available across different devices. As a proof of concept, the energy efficient sensing library was built, which embodies part of the ideas presented in this dissertation.