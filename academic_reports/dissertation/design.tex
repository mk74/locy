\section{Design}
\label{s:design}
The section describes the design of the software components and the experiments involved in the dissertation. The first subsection outlines how energy demands of different sensors are compared. This consists of the design principles behind sample sensor applications and the details on how to measure their energy efficiency. The second subsection layouts the energy efficient sensing library, Sensy. [XXX keep going here]

\subsection{Sensors energy measurements}
\label{s:design:measurements}
Sensors energy measurements is a part of the project, where I determine how energy efficient different sensors are. To achieve this aim, the series of simple Android applications, \textbg{Sample Sensor Applications},  is created. Each of those applications keeps continuously sampling one sensor and switch off all others. Sample Sensor Applications are designed in the way that only one difference between them should be which sensor is being sampled. To compare sensors' energy demands, the method of measuring energy efficiency of the applications is being proposed. The method calculates time which applications need to deplete one percentage of battery life. The order of energy efficiency of different sensors may be established by comparing those time measurements. The most energy efficient sensor will be the one which one percentage battery depletion takes the longest. 

\subsubsection{Sample sensor applications}
\label{s:design:measurements:sampleapps}
To determine differences in sensors' energy efficiency, each of sample applications keep sampling only one sensor while all others sensors are being switched off. To only study the energy efficiency of sampling, we try to minimize the impact of other issues such as different sampling frequencies of sensors and various sensors' output.

Sampling frequencies vary among sensors. Inertial sensors e.g., accelerometer or gyroscope could be sampled as often as every 20 milliseconds, whereas the minimum frequency of GPS equals to 20 seconds. The issue is more complex in case of wireless communication sensors such as Bluetooth or IEEE 802.11. A full Bluetooth scan takes around 12 secs, but it could be stopped earlier and started again. To alleviate those differences, the least energy-efficient strategy of complete, correct sampling is chosen for the comparison of sensors' energy efficiency. Sensors are being sampled as often as possible providing that the previous sample delivered valid data. For example, the next IEEE 802.11 scan will be started once the previous one has been finished\ (IEEE 802.11 scan takes around 2.5 secs depending on the device). This provides complete information on available access points while minimizing sampling frequency, which increases its energy consumption. The strategy will look similar for GPS and Bluetooth. In case of other sensors (inertial sensors, camera and microphone), the strategy will be executed by sensors being sampled as often as possible (sampling is immediate). 


Sensors also deliver different sizes of raw data. The result of wireless communication scans\ (IEEE 802.11 and Bluetooth) is the list of records (e.g., 20 hot spots with their names and other parameters), whereas inertial navigation is three coordinates. More complex data is also delivered by camera or microphone (XXX not finished!!)
					
		-standardized output\\
				-its frequency\\
					results will come on different frequencies\\
					but they are printed on the same frequency\\
						even if there are no new results\\
				-how it looks like\\
					WiFi Scan results look different (20 records) than light sensor (1 int)\\
					solution:\\
						standardized to "values: int int int" or "values: int"
							e.g. Wireless scanning(WiFi, Bluetooth, Bluetooth LTE) only shows number of records\\
							show example\\	
		-conclusion: hope to only measure what we want to measure\\				
				
\subsubsection{1\% battery depletion}	
\label{s:design:measurements:method}			
how to compare:\\
	measure how long it takes to deplete 1% of battery life\\
		measure on exactly the same percentage across different apps\\
		make a couple of samples, average and compare\\
		reasons:\\
			people measure the whole battery life, but takes too long\\
			
			
		predictions:\\
			the battery life is nonlinear, but we hope that the behaviour on the the same percentage will be the same.\\
			we hope that those measurements will be repetitive and similar.\\
			measurement will not take long\\
			1% is accuracy which Android API gives\\
	Trepn Profiler\\
			
			
		
		
		
\subsection{Sensy}
\label{s:design:sensy}
