% $Id: macros.tex 391 2008-06-05 13:45:52Z tristan $

\usepackage{pslatex}
\usepackage{subfigure}
\usepackage[usenames]{color}

% any local macros can go here %



%todo boxes for things.
 \ifx\draftmode\defined
  \newcommand{\todobox}[1]{}%
  \else
 \newcommand{\todobox}[1]{%
     \centering
		\fbox{\parbox[l]{\columnwidth}{\textcolor{blue}{TODO: #1}}}\\%
		%\framebox[100mm][h]{\textcolor{blue}{TODO: #1}}\\%
 }
 \fi 
 
 %reasoning boxes for things.
 \ifx\draftmode\defined
  \newcommand{\myreason}[1]{}%
  \else
 \newcommand{\myreason}[1]{%
     \centering
		\fbox{\parbox[l]{\columnwidth}{\textcolor{green}{REASONING: #1}}}\\%
		%\framebox[100mm][h]{\textcolor{blue}{TODO: #1}}\\%
 }
 \fi 
 

 %simply type \mycomment{} to get a comment box
  \ifx\draftmode\undefined
  \newcommand{\mycomment}[1]{}
  \else
 \newcommand{\mycomment}[1]{% 
            \centering
		 	\fbox{\parbox[l]{\columnwidth}{\textcolor{red}{COMMENT: #1}}}\\%
  }
  \fi 

 

% include a figure from the plots/ subdirectory
% \plot{file} will read captionm from plots/file.tex and graphic
% from plots/file.{eps,pdf}
\newcommand{\plot}[1]{%
\begin{figure}[tbp]
    \centerline{\resizebox{0.8\linewidth}{!}{\includegraphics{plots/#1}}}
    \caption{\label{p:#1}\protect\input{plots/#1}}
\end{figure}
}

% plot that spans both columns
\newcommand{\plotwide}[1]{%
\begin{figure*}[htbp]
    \centerline{\resizebox{0.75\linewidth}{!}{\includegraphics{plots/#1}}}
    \caption{\label{p:#1}\protect\input{plots/#1}}
\end{figure*}
}

% compare two plots
% \plot{file1,file2,caption} will show file1 and file2 side-by-side
% caption will be used for the overall caption
\newcommand{\plots}[3]{%
\begin{figure*}[tbp]
    \centering
        \label{p2:#1}
        \subfigure[{\protect\input{plots/#1}}]{
            \label{p:#1}%
            \includegraphics[width=0.34\textwidth]{plots/#1}%
        }
        \hspace{0.15\textwidth}%
        \subfigure[{\protect\input{plots/#2}}]{
            \label{p:#2}%
            \includegraphics[width=0.34\textwidth]{plots/#2}%
        }
        \caption{\protect{#3}}
\end{figure*}
}

\ifx\draftmode\undefined
% For final version
\newcommand{\comment}[1]{}%
\else
\ifpdf
% For draft mode
% ACM format doesn't like margin pars
%\newcommand{\comment}[1]{ \marginpar{$\leftarrow$}{\bf $<$#1$>$} }
\newcommand{\comment}[1]{ {\textcolor{red}{\bf $<$#1$>$}} }
\else
\newcommand{\comment}[1]{{\textcolor{red}{ \bf $<$#1$>$}}}
\fi
\fi

% Referring to a plot:
\newcommand{\plotref}[1]{Figure~\ref{p:#1}}
% or two plots:
\newcommand{\plotrefs}[2]{Figures~\ref{p:#1} and \ref{p:#2}}
% or a range of plots:
\newcommand{\plotrange}[2]{Figures~\ref{p:#1}--\ref{p:#2}}
% or a pair of plots (using \plots macro)
\newcommand{\plotsref}[1]{Figure~\ref{p2:#1}}

\newcommand{\compareplotref}[1]{Figure~\ref{p2:#1}}
\newcommand{\compareplotrefs}[2]{Figures~\ref{p2:#1} and \ref{p2:#2}}
\newcommand{\compareplotrange}[2]{Figures~\ref{p2:#1}--\ref{p2:#2}}