\section{Requirements specification}
\label{s:requirements}
\hspace{10pt} This section specifies the properties the software solution must satisfy to fulfill the objectives of the project. Analogically to the objectives, the requirements are characterized by priority: primary\ (priority 1), secondary\ (2) and tertiary requirements\ (3). Furthermore, the requirements are separated into two sections.

\subsection{Requirements: mobile phones and energy measurements}

\begin{table}[H]
	\centering
    \begin{tabular}{| l | p{12.0cm} |}
    \hline
     \multicolumn{2}{| l |}{ \textbf{1. All software components of the project will work on the Android mobile operating system.} }  \\ \hline
    \bf{Type} & non-functional\\ \hline
    \bf{Priority} & 1\\ \hline
    \bf{Description} & The energy-efficient sensing library, Locy and all Sample Sensor Applications used for energy measurements should work on the Android mobile operating system. \\ \hline
    \end{tabular}
    \label{r:devices:android}
\end{table}

\begin{table}[H]
	\centering
    \begin{tabular}{| l | p{12.0cm} |}
    \hline
     \multicolumn{2}{| l |}{ \textbf{2. Software components should work on a variety of Android devices.} }  \\ \hline
    \bf{Type} & non-functional\\ \hline
    \bf{Priority} & 2\\ \hline
    \bf{Description} & The energy-efficient sensing library, Locy and all Sample Sensor Applications used for energy measurements should work on a variety of Android devices.\\ \hline
    \end{tabular}
     \label{r:devices:different}
\end{table}

\begin{table}[H]
	\centering
    \begin{tabular}{| l | p{12.0cm} |}
    \hline
     \multicolumn{2}{| l |}{ \textbf{3. The project should empirically determine the energy costs of different sensors.} }  \\ \hline
    \bf{Type} & functional\\ \hline
    \bf{Priority} & 1\\ \hline
    \bf{Description} & In the experiments, the order of different sensors' energy efficiency should be established. The Sample Sensor Applications and a measurement method are involved in this requirement.\\ \hline
    \end{tabular}
    \label{r:measurement:main}
\end{table}


\begin{table}[H]
	\centering
    \begin{tabular}{| l | p{12.0cm} |}
    \hline
     \multicolumn{2}{| l |}{ \textbf{4. The library will calculate energy costs of different sensors in real time.} }  \\ \hline
    \bf{Type} & functional\\ \hline
    \bf{Priority} & 2\\ \hline
    \bf{Description} & The energy measurements of the sensors could be calculated online, once the library is installed on a user's device. \\ \hline
    \end{tabular}
    \label{something4}
    \label{r:measurement:online}
\end{table}

\subsection{Requirements: the energy-efficient sensing library}
\begin{table}[H]
	\centering
    \begin{tabular}{| l | p{12.0cm} |}
    \hline
     \multicolumn{2}{| p{14.0cm} |}{ \textbf{5. Energy efficiency of the library should be tested against a baseline implementation.} }  \\ \hline
    \bf{Type} & functional\\ \hline
    \bf{Priority} & 1\\ \hline
    \bf{Description} & In another series of experiments, the power consumption of the smart library needs to be compared against a naive implementation. This allows the library's utility to be validated. \\ \hline
    \end{tabular}
    \label{r:library:evaluation}
\end{table}

\begin{table}[H]
	\centering
    \begin{tabular}{| l | p{12.0cm} |}
    \hline
     \multicolumn{2}{| p{14.0cm} |}{ \textbf{6. The library will be integrated with Tristan's experience sampling method~ (ESM) mobile application.} }  \\ \hline
    \bf{Type} & functional\\ \hline
    \bf{Priority} & 1\\ \hline
    \bf{Description} & The library should be easily plugged into existing mobile applications. Tristan's ESM is used as an example application for that purpose.\\ \hline
    \end{tabular}
    \label{r:library:esm}
\end{table}

\begin{table}[H]
	\centering
    \begin{tabular}{| l | p{12.0cm} |}
    \hline
     \multicolumn{2}{| l |}{ \textbf{7. The energy saving algorithm should be adaptive according to current battery life.} }  \\ \hline
    \bf{Type} & functional\\ \hline
    \bf{Priority} & 2\\ \hline
    \bf{Description} & The algorithm should be more energy-efficient and less accurate while battery level is low. This would allow to utilize better the remains of the battery life.\\ \hline
    \end{tabular}
    \label{r:library:adaptive}
\end{table}

\begin{table}[H]
	\centering
    \begin{tabular}{| l | p{12.0cm} |}
    \hline
     \multicolumn{2}{| l |}{ \textbf{8. The library intends to manage mobile applications' contention for sensor usage.} }  \\ \hline
    \bf{Type} & functional\\ \hline
    \bf{Priority} & 2\\ \hline
    \bf{Description} & Many application can use a library in energy-efficient manner. That would make a library more practical for industry purposes.\\ \hline
    \end{tabular}
    \label{r:library:contention}
\end{table}

\begin{table}[H]
	\centering
    \begin{tabular}{| l | p{12.0cm} |}
    \hline
     \multicolumn{2}{| p{14.0cm} |}{ \textbf{9. The energy saving algorithm should improve its performance by learning from user behaviour. } }  \\ \hline
    \bf{Type} & functional\\ \hline
    \bf{Priority} & 3\\ \hline
    \bf{Description} & A library may improve its energy efficiency while regular patterns in a user behaviour are known. The utility if the library would be higher in users' eyes and it is considered as original idea in research community.\\ \hline
    \end{tabular}
    \label{r:library:history}
\end{table}