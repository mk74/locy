\documentclass[a2,landscape]{a0poster}
\usepackage{mathptmx}
\usepackage[scaled=.90]{helvet}
\usepackage{courier}
\usepackage{postercols}
\usepackage{flowfram}
\usepackage{graphicx}
\setlength{\vcolumnsep}{\baselineskip}
\setlength{\columnsep}{\vcolumnsep}
\Ncolumntop{static}{3}{2.5in}
\setstaticframe{1}{label={title}}
\newlength\offset
\setlength{\offset}{5in}
\addtolength{\offset}{\vcolumnsep}

%\computeflowframearea{2,3}
%\addtolength{\ffareaheight}{-\offset}
%\setflowframe{2,3}{y=\offset,height=\ffareaheight}
%\newstaticframe{\ffareawidth}{5in}{\ffareax}{0in}[table]

\setstaticframe{2}{clear}
\setallflowframes{border=plain}
\setallstaticframes{border=plain}
\title{Locy: Energy-efficient sensing with Android smartphones.}
\author{Martin Kukla (Supervisor: Dr Tristan Henderson)}
\date{}
\begin{document}
\begin{staticcontents*}{title}
\maketitle	 
\end{staticcontents*}
\thispagestyle{empty}

\setlength{\tabcolsep}{6pt}

\begin{center}
\section*{Introduction}
\end{center}
		
\begin{itemize}
   \item Phone sensing may be utilized by mobile applications to provide \textbf{advanced services} such as navigation systems.
   
\begin{center}
	\includegraphics[scale=0.08]{plots/logo_yelp}   
	\includegraphics[scale=0.35]{plots/logo_apple_maps}
   \includegraphics[scale=0.08]{plots/logo_google_maps}
	\includegraphics[scale=0.3]{plots/logo_facebook}
	\includegraphics[scale=0.4]{plots/logo_mhealth}\\
	\includegraphics[scale=0.4]{plots/logo_foursquare}
	\includegraphics[scale=1.2]{plots/logo_google_now}
\end{center}
	
   \item \textbf{Phone sensing} fetches raw sensor data\ (e.g. from an accelerometer) and tries to extract high-level information from it\ (e.g. a user is walking).
   \item Such a process may have \textbf{high energy demands}, which is crucially important to mobile phone users.
   \item To solve the problem:
	   \begin{itemize}
   			\item investigate many devices.
   			\item establish the energy efficiency of their sensors.
   			\item leverage results for energy-efficient sensing.
   			\item build \textbf{Locy}, an energy efficient sensing library.
	   \end{itemize}
  \end{itemize}
  
\begin{center}
\includegraphics[scale=0.7]{plots/low_battery}
\includegraphics[scale=0.7]{plots/sad_face}
\end{center}

\mbox{}\framebreak
\begin{center}
\section*{Solution}
\end{center}
\begin{itemize}
   \item Energy efficiency of sensors is \textbf{different among the devices}.
   
%\begin{figure}[H]
\includegraphics[scale=0.9]{plots/shared}
%\caption{\label{p:shared} \footnotesize{Energy efficiency of sensors differs depending on a device.} }
%\end{figure}

   \item For all mobile phones, \textbf{accelerometer is more energy-efficient} than the standard localization sensors.

%\begin{figure}[H]
\includegraphics[scale=0.8]{plots/acc_vs_loc}
%\caption{\label{p:acc_vs_loc} \footnotesize{Accelerometer is more energy-efficient than the standard localization sensors.} }
%\end{figure}

   \item Locy
   \item movement detection which leverages energy-efficient accelerometer to switch off GPS [MAYBE GRAPH]
   \item duty-cycling + adaptive towards the battery life
  \end{itemize}


\mbox{}\framebreak
\begin{center}
\section*{Evaluation}
\end{center}
\begin{itemize}
   \item the \textbf{first scenario}:
   
%\begin{figure}[H]
\includegraphics[scale=0.65]{plots/locy_eval_inplace}
%\caption{\label{p:locy_eval_place} \footnotesize{Locy is more energy-efficient than the naive GPS localization while a user is in place.} }
%\end{figure}
XXX explain 

   \item the \textbf{second scenario}:
   
%\begin{figure}[H]
\includegraphics[scale=0.65]{plots/locy_eval_other}
%\caption{\label{p:locy_eval_other} \footnotesize{Locy is more energy-efficient than the naive GPS localization while a user is half of the time moving and the rest he is staying in one place.} }
%\end{figure}
XXX explain 

  \end{itemize}

\begin{center}
\section*{Conclusions}
\end{center}

\textbf{Locy} is more energy-efficient than the standard Android implementation. 
\begin{center}
\includegraphics[scale=0.7]{plots/full_battery}
\includegraphics[scale=0.7]{plots/happy_face}
\end{center}


\end{document}
